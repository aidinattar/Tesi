% !TEX root = ../../Tesi_Triennale_PMNS.tex
\chapter{Conclusioni}
\label{chapter:conclusioni}
La rivelazione del primo segnale di onda gravitazionale da un sistema binario di stelle di neutroni nell'Agosto 2017 ha aperto un nuovo ramo di studio della fisica, in particolare per le equazioni di stato delle stelle di neutroni. Dopo la coalescenza, la natura del corpo celeste rimanente è determinata in modo primario dalle masse delle stelle progenitrici e dall'equazione di stato della materia nucleare e questo determina il segnale di post-coalescenza. Purtroppo, a causa della limitata sensibilità dei rivelatori, questa fase del segnale non è stata ancora identificata, nè nel primo segnale rivelato, nè in quello successivo, non permettendo di determinare quale, tra i diversi modelli proposti per le NS, sia quello corretto.

Questa tesi si propone di valutare la capacità di ricostruzione prevista per il run O4, andando a simulare un grande numero di eventi con posizione e polarizzazione casuali e calcolando alcune grandezze che permettono di esprimere quantitativamente l'efficienza dell'algoritmo. In particolare le curve di overlap e di energia residua permettono di fare la caratterizzazione.
La seconda parte dell'analisi si è invece concentrata sulla caratterizzazione del segnale di post-coalescenza che, con il procedimento descritto, viene isolato permettendo di valutarne il contributo in energia, la frequenza pesata e la larghezza di banda pesata. I risultati mostrano che il metodo utilizzato porta a una sottostima sistematica della frequenza, poiché nella media viene considerato anche il contributo della parte finale dello spiraleggiamento, a frequenze significativamente inferiori rispetto alla post-coalescenza. Inoltre il metodo utilizzato non permette di distinguere in modo automatico una coalescenza con stelle progenitrici tali da portare a un buco nero o a stelle di neutroni stabili o instabili. Sarà necessario quindi implementare delle statistiche che permettano di discernere il tipo di segnale che si sta valutando.