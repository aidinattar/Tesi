% !TEX root = ../../Tesi_Triennale_PMNS.tex
\chapter{Conclusioni}
\label{chapter:conclusioni}
GW170817 è il primo  segnale di onda gravitazionale dovuto alla coalescenza di un sistema binario di stelle di neutroni rivelato. Questo segnale gravitazionale è stato inoltre il primo ad essere rivelato in  coincidenza temporale con l'osservazione di un segnale di lampo gamma, rappresentando così un punto di partenza per l'astronomia multimessaggera. È inoltre interessante perché lo studio del segnale gravitazionale può fornire utili informazioni sulla natura delle stelle di neutroni e in particolare sulle loro equazioni di stato. Questo lavoro di tesi si è concentrato sullo studio della fase di post-coalescenza del segnale che dovrebbe permettere di ottenere informazioni sulla natura del corpo celeste rimanente,  determinata in modo primario dalle masse delle stelle progenitrici e dall'equazione di stato della materia nucleare delle stelle di neutroni. La fase di post-coalescenza di GW170817 non è stata rivelata, a causa della limitata sensibilità della rete di interferometri, non permettendo di determinare quale tra i diversi modelli proposti per le stelle di neutroni sia quello corretto.

Questa tesi si propone di effettuare un'analisi tramite eventi simulati per caratterizzare la capacità dell'algoritmo Coherent WaveBurst di ricostruire questo tipo di segnale, utilizzando una procedura di caratterizzazione sviluppata in \cite{Puecher_2018} e in \cite{Tringali_2017}.

Si studia in questo modo il segnale di post-coalescenza, in una configurazione non testata precedentemente, utilizzando tre equazioni di stato, APR4, SHT2.0 e SHT2.2, con stati finali diversi tra loro e secondo i criteri descritti nella tesi. 
La valutazione della capacità di ricostruzione dell'algoritmo con questa configurazione, combinando il rapporto segnale su rumore ricostruito e iniettato costruendo estimatori statistici, mostra buoni risultati coerenti con le analisi precedenti. Anche la valutazione della frequenza pesata e la larghezza di banda pesata mostra i risultati attesi. \\
%Il metodo utilizzato per studiare separatamente il segnale di post-coalescenza non è stato implementato in tutte le sue parti ma prevede solo la stima dell'energia della post-coalescenza e la frequenza e larghezza di banda pesate
L'analisi riportata in questa tesi ha incluso lo studio dell'energia della post-coalescenza e la frequenza e larghezza di banda pesate; questi estimatori non bastano tuttavia per discernere eventi in cui vi è post-coalescenza e in cui non vi è, e lo sviluppo naturale di questo lavoro di tesi sarebbe implementare le tecniche utilizzate in \cite{Puecher_2018} in questa nuova configurazione.\\
Un ulteriore sviluppo di questo lavoro di tesi sarebbe la ricerca di una nuova configurazione più adatta allo studio della EOS APR4, che per questo lavoro ha richiesto il taglio di parte del segnale, che in particolare si riferisce spesso al segnale di post-coalescenza. Il tentativo fatto a fine tesi di portare la separazione massima in frequenza nella mappa tempo-frequenza del segnale da 128Hz a 512Hz non ha mostrato evidenti miglioramenti, le prossime analisi dovrebbero aumentare ulteriormente questa separazione e, possibilmente, aumentare anche la soglia di probabilità per la selezione dei pixel, andando tuttavia ad aumentare la probabilità di falso allarme dell'algoritmo.


%Questa tesi si propone di valutare la capacità di ricostruzione prevista per il run O4, andando a simulare un grande numero di eventi con posizione e polarizzazione casuali e calcolando alcune grandezze che permettono di esprimere quantitativamente l'efficienza dell'algoritmo. In particolare le curve di overlap e di energia residua permettono di farne una caratterizzazione.\\
%La seconda parte dell'analisi si è invece concentrata sulla caratterizzazione del segnale di post-coalescenza che con il procedimento descritto viene isolato, permettendo di valutarne il contributo in energia, la frequenza pesata e la larghezza di banda pesata. I risultati mostrano che il metodo utilizzato porta a una sottostima sistematica della frequenza, poiché nella media viene considerato anche il contributo della parte finale dello spiraleggiamento, a frequenze significativamente inferiori rispetto alla post-coalescenza. Inoltre il metodo utilizzato non permette di distinguere in modo automatico una coalescenza con stelle progenitrici tali da portare a un buco nero o a stelle di neutroni stabili o instabili. Sarà necessario quindi implementare delle statistiche che permettano di discernere il tipo di segnale che si sta valutando.

