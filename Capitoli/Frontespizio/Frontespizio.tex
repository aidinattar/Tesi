% !TEX root = ../../Tesi_Triennale_PMNS.tex
\begin{titlepage}
	\vspace{5mm}
	\begin{figure}[hbtp]
		\centering
		\includegraphics[scale=.13]{figures/Frontespizio/UNIPD.png}
	\end{figure}
	
	\vspace{5mm}
	
	\begin{center}
		{{\huge{\textsc{\bf UNIVERSIT\`A DEGLI STUDI DI PADOVA}}}\\}
		\vspace{5mm}
		{\Large{\bf Dipartimento di Fisica e Astronomia ``Galileo Galilei''}} \\
		\vspace{5mm}
		{\Large{\textsc{\bf Corso di Laurea in Fisica}}}\\
		\vspace{20mm}
		{\Large{\textsc{\bf Tesi di Laurea}}}\\
		\vspace{30mm}
		\begin{spacing}{3}
			{\LARGE \textbf{Ricerca di segnali gravitazionali dovuti alla coalescenza di sistemi binari di stelle di neutroni nella fase di post coalescenza}}\\
		\end{spacing}
		\vspace{8mm}
	\end{center}
	
	\vspace{20mm} %era 20
	
	\begin{spacing}{2}
		\begin{tabular}{ l  c  c c c  cc c c c c  l }
			{\Large{\bf Relatrice}} &&&&&&&&&&& {\Large{\bf Laureando}}\\
			{\Large{\bf Dott.ssa Claudia Lazzaro}} &&&&&&&&&&& {\Large{\bf Aidin Attar}}\\
			%{\Large{\bf Correlatore}}\\
			%{\Large{\bf Prof./Dr. Nome Cognome}}\\
		\end{tabular}
	\end{spacing}
	\vspace{15 mm} %era 15
	
	\begin{center}
		{\Large{\bf Anno Accademico 2020/2021}}
	\end{center}
\end{titlepage}
\clearpage{\pagestyle{empty}\cleardoublepage}
%\thispagestyle{empty}
%\vspace*{\stretch{1}}
%\begin{flushright}
%	\itshape Alla mia famiglia\\
%	e Aidin
%\end{flushright}
%\vspace{\stretch{2}}
%\clearpage
%\chapter*{Abstract}

\tableofcontents
\newpage
\thispagestyle{plain}
\begin{Abstract}
	\begin{changemargin}{1cm}{1cm}
		La tesi è dedicata allo studio dei segnali di onda gravitazionale generati dalla post-coalescenza di sistemi binari di stelle di neutroni attraverso l'algoritmo Coherent WaveBurst.
		Nella prima parte si descrive il fenomeno dello spiraleggiamento di un sistema binario di stelle di neutroni con la conseguente emissione di onde gravitazionali e si studiano gli stati finali in seguito alla coalescenza, con particolare attenzione ai segnali di onda gravitazionale. Nella seconda parte si ripercorre la prima osservazione di onda gravitazionale da questo tipo di sistemi e i limiti sperimentali che non permettono l'osservazione del segnale di post-coalescenza. Infine si fa una analisi delle curve di sensibilità dell'algoritmo che si utilizza e ...
	\end{changemargin}
\end{Abstract}
